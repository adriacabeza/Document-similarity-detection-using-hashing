\documentclass[12pt]{article}
\usepackage[utf8]{inputenc}
\usepackage{cite}
\usepackage{float}
\usepackage{listings}
\usepackage[scale=0.75]{geometry}
\usepackage[hidelinks]{hyperref}
\usepackage{graphicx}
\usepackage{mathtools}
\usepackage{xcolor} % for setting colors
\author{Carlos Bergillos, Antoni Rambla, Adrià Cabeza\\ Departament de Computació}
\title{Compiling and executing manual}


\begin{document}
  \maketitle  
   \newpage
   \section{Compiling}
   To make easier the work of compiling end executing the source files we have created a makefile that can be seen in our source folder. Inside the makefile we have determined several functionalities that can help us in the task of compiling our source files:
   
   \begin{itemize}
   \item \textbf{Default option}: it compiles all our source files and generates all our executables. It also creates several object files in order to be able to generate some of our executables. 
   To run it you have to type: \textit{make}
   \item \textbf{Delete all the object and executables files} generated in the folder:
   To run it you have to type: \textit{make clean}
   \end{itemize}
      \section{Executing}
         After using the makefile to compile our files we can execute them.
            
   \begin{itemize}
           \item Execute the program that \textbf{compares two different documents using Jaccard Similarity}:
   To run it you have to type: \textit{./compjac}
  

   \item Execute the program that generates \textbf{permutations} of a file:
   To run it you have to type: \textit{./permutacions}
    
\item Execute the program that creates the data for our \textbf{Performance of Different Hashing Algorithms} experiment:
   To run it you have to type: \textit{./jocProvesHashTimes}
  
\item Execute the program that creates the data for our \textbf{Performance of Different Document Similarity Approaches} experiment:
   To run it you have to type: \textit{./jocProvesJaccSim}
   \item Execute the program that creates the data for our \textbf{Precision of Jaccard Similarity Approximations} experiment:
   To run it you have to type: \textit{./jocProvesJaccSimLsh}
   
      \item Execute the program that creates the data for our \textbf{k-shigles size} experiment:
   To run it you have to type: \textit{./jovProvesJaccard}

   \item Execute the program where you can try all of our implemented functionalities by answering the questions that are prompted:
   To run it you have to type: \textit{./mainDriver}
      \end{itemize}
\end{document}

