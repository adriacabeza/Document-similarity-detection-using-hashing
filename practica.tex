\documentclass[12pt]{article}
\usepackage[utf8]{inputenc}
\author{Carlos Bergillos, Antoni Ramble, Adrià Cabeza}
\title{Document similarity detection using hasing }
\date{\today}


\begin{document}
  \maketitle  

  
	\begin{abstract}
ESTO ES UNA PRUEBA PARA VER COMO QUEDA n this experiment we studied a very important physical effect by measuring the
dependence of a quantity $V$ of the quantity $X$ for two different sample
temperatures.  Our experimental measurements confirmed the quadratic dependence
$V = kX^2$ predicted by Someone's first law. The value of the mystery parameter
$k = 15.4\pm 0.5$~s was extracted from the fit. This value is
not consistent with the theoretically predicted $k_{theory}=17.34$~s. We attribute this
discrepancy to low efficiency of our $V$-detector.
\end{abstract}

\newpage
\tableofcontents
\newpage

\section{Introduction}

Minhashing: compresses large sets in such a wat that we can deduce the similatity of the underlying sets from their compressed versions. 
When we search for similar items of any kind there may be far too many pairs of items to test each pair for their degree of similarity

\section{Theory}

First we have to focus into the definition of similarity, when we talk about the "Jaccard similarity",which is calculated by looking at the relative size of their intersection. 

The Jaccard similarity, also known as Jaccard index is a statistical measure of similarity of sets. For two sets, it is defined as the size of the intersection divided by the size of the union of the sample sets:
\medbreak
\centerline{\small $J(A,B)=\frac{\left |A\cap B  \right |}{ \left |A\cup B  \right |} = \frac{\left |A\cap B  \right |}{ \left|A\right|+\left|B\right|-\left |A\cap B  \right |} $}



\begin{thebibliography}{100}

\bibitem{Similarity}
J.\ Bank and B.\ Cole,
\textit {Calculating the Jaccard similarity coefficient with map reduce for entity pairs in
Wikipedia} (Wikipedia Similarity Team, 2008).

\bibitem{Cambridge}
J.\ Leskovec, A.\ Rajaraman and J.\ Ullman
\textit{ Finding Similar Items. In Mining of Massive Datasets} (Cambridge University Press, 2014).



\end{thebibliography}

\end{document}

